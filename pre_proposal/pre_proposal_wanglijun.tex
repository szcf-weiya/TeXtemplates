\documentclass{beamer}

%\usepackage[british]{babel}
\usepackage[american]{babel}

\usepackage{graphicx,hyperref,ru,url}
\usepackage{fontspec}
\usepackage[T1]{fontenc}
\usefonttheme[onlymath]{serif}
%\usepackage{cmbright}
%\renewcommand\sfdefault{cmbr}
\usepackage{xeCJK}
\usepackage{indentfirst}
\usepackage{longtable}
\usepackage{graphicx}
\usepackage{float}
\usepackage{rotating}
\usepackage{subfigure}
\usepackage{tabu}
\usepackage{amsmath}
\usepackage{amssymb}
\usepackage{bm}
\usepackage{setspace}
\usepackage{amsfonts}
\usepackage{appendix}
\usepackage{listings}
\usepackage{multicol}
 \usepackage{ulem}
\usepackage{xcolor}
\usepackage{geometry}
\setCJKfamilyfont{cjkhwxk}{STXINGKA.TTF}
\newcommand*{\cjkhwxk}{\CJKfamily{cjkhwxk}}
%\newfontfamily{\consolas}{Consolas}
%\newfontfamily{\monaco}{Monaco}
%\setmonofont[Mapping={}]{Consolas}	%英文引号之类的正常显示,相当于设置英文字体
%\setsansfont{Consolas} %设置英文字体 Monaco, Consolas,  Fantasque Sans Mono
%\setmainfont{Times New Roman}
\newfontfamily{\consolas}{YaHeiConsolas.ttf}
\newfontfamily{\monaco}{MONACO.TTF}
\newfontfamily{\ubuntumonoB}{UbuntuMono-B.ttf}
\newfontfamily{\ubuntumonoBI}{UbuntuMono-BI.ttf}
\newfontfamily{\ubuntuR}{Ubuntu-R.ttf}
\newfontfamily{\ubuntuC}{Ubuntu-C.ttf}
\setCJKmainfont{STZHONGS.TTF}
%\setmainfont{MONACO.TTF}
%\setsansfont{MONACO.TTF}
%\setsansfont{UbuntuMono-R.ttf}
%\setsansfont{UbuntuMono-B.ttf}
\newcommand{\verylarge}{\fontsize{60pt}{\baselineskip}\selectfont}  
\newcommand{\chuhao}{\fontsize{44.9pt}{\baselineskip}\selectfont}  
\newcommand{\xiaochu}{\fontsize{38.5pt}{\baselineskip}\selectfont}  
\newcommand{\yihao}{\fontsize{27.8pt}{\baselineskip}\selectfont}  
\newcommand{\xiaoyi}{\fontsize{25.7pt}{\baselineskip}\selectfont}  
\newcommand{\erhao}{\fontsize{23.5pt}{\baselineskip}\selectfont}  
\newcommand{\xiaoerhao}{\fontsize{19.3pt}{\baselineskip}\selectfont} 
\newcommand{\sihao}{\fontsize{14pt}{\baselineskip}\selectfont}      % 字号设置  
\newcommand{\xiaosihao}{\fontsize{12pt}{\baselineskip}\selectfont}  % 字号设置  
\newcommand{\wuhao}{\fontsize{10.5pt}{\baselineskip}\selectfont}    % 字号设置  
\newcommand{\xiaowuhao}{\fontsize{9pt}{\baselineskip}\selectfont}   % 字号设置  
\newcommand{\liuhao}{\fontsize{7.875pt}{\baselineskip}\selectfont}  % 字号设置  
\newcommand{\qihao}{\fontsize{5.25pt}{\baselineskip}\selectfont}    % 字号设置 

\definecolor{cred}{rgb}{0.8,0.8,0.8}
\definecolor{cgreen}{rgb}{0,0.3,0}
\definecolor{cpurple}{rgb}{0.5,0,0.35}
\definecolor{cdocblue}{rgb}{0,0,0.3}
\definecolor{cdark}{rgb}{0.95,1.0,1.0}
\lstset{
	numbers=left,
	numberstyle=\tiny\color{black},
	showspaces=false,
	showstringspaces=false,
	basicstyle={\footnotesize}\consolas\itshape,
	keywordstyle=\color{red}\bfseries,
	commentstyle=\color{cgreen},
	stringstyle=\color{cred},
	frame=lines,
	escapeinside=``,
	xleftmargin=1em,
	xrightmargin=1em, 
	%backgroundcolor=\color{cdark},
	aboveskip=1em,
	breaklines=true,
	tabsize=4
} 

\usepackage[style=apa,backend=biber,sortcites=true,citestyle=authoryear,natbib]{biblatex}
\DeclareLanguageMapping{american}{american-apa}
\DeclareCiteCommand{\cite}
{\color{blue}\usebibmacro{prenote}}%
{\usebibmacro{citeindex}%
	\usebibmacro{cite}}
{\multicitedelim}
{\usebibmacro{postnote}}

\DeclareCiteCommand{\citet}
{\color{blue}\usebibmacro{prenote}}%
{\usebibmacro{citeindex}%
	\usebibmacro{cite}}
{\multicitedelim}
{\usebibmacro{postnote}}

\DeclareCiteCommand{\parencite}[\mkcolorbibparens]
{\usebibmacro{prenote}}%
{\usebibmacro{citeindex}%
	\usebibmacro{cite}}
{\multicitedelim}
{\usebibmacro{postnote}}

\makeatletter
\newrobustcmd{\mkcolorbibparens}[1]{%
	\begingroup
	\color{blue}%
	\blx@blxinit
	\blx@setsfcodes
	\bibopenparen#1\bibcloseparen
	\endgroup}
\makeatother
\bibliography{ref.bib}
\addbibresource{ref.bib}
% The title of the presentation:
%  - first a short version which is visible at the bottom of each slide;
%  - second the full title shown on the title slide;
\title[动态因子模型在高维数据中的应用]{
  \xiaoerhao 动态因子模型在高维数据中的应用}

% Optional: a subtitle to be dispalyed on the title slide
\subtitle{\sihao\cjkhwxk 有监督的动态因子模型预测}

% The author(s) of the presentation:
%  - again first a short version to be displayed at the bottom;
%  - next the full list of authors, which may include contact information;
\author[汪利军]{
  汪利军(导师:张荣茂)}

% The institute:
%  - to start the name of the university as displayed on the top of each slide
%    this can be adjusted such that you can also create a Dutch version
%  - next the institute information as displayed on the title slide
\institute[\xiaosihao\fontspec{LHANDW.TTF}Zhejiang University]{\sihao {数学科学学院}}

% Add a date and possibly the name of the event to the slides
%  - again first a short version to be shown at the bottom of each slide
%  - second the full date and event name for the title slide
\date[\today]{
  \today \\
  }
%\graphicspath{{img/}}

\begin{document}

\begin{frame}
  \titlepage
\end{frame}




\begin{frame}
  \frametitle{内容概要}

  \tableofcontents
\end{frame}

% Section titles are shown in at the top of the slides with the current section 
% highlighted. Note that the number of sections determines the size of the top 
% bar, and hence the university name and logo. If you do not add any sections 
% they will not be visible.

\section{研究背景}
\begin{frame}{研究背景}
	\begin{itemize}
		\setlength\itemsep{1.2em}
		\item 	在宏观经济学中,用于预测的时间序列的个数$N$非常大,经常远大于观测值个数$T$,即$N >> T$。这种高维问题可以用少量的潜在因子来建模,采用动态因子模型能够起到降维目的\parencite{stock2002forecasting};
		\item 估计动态因子模型中的参数有多种方法,如传统的主成分方法(PC)\parencite{stock2002forecasting},以及更适用于异方差情形的广义最小二乘(GLS)\parencite{breitung2011gls}。
	\end{itemize}
%数据集非常大,包含了高维变量。举个例子\parencite{fan2011high}
	%\begin{itemize}
	%	\item 世界银行(World Bank)有40年200个国家的数据,在投资组合配置中,股票的个数成千上万,原高于样本大小。
	%	\item 对每个地区的房价进行建模时,地区的个数可以成千上万,但样本大小仅仅240个月份(或者20年)。
	%\end{itemize}
\end{frame}

\section{数学定义}

%\begin{frame}{时间序列向量}
%	对于$n\times 1$的时间序列向量$X^*_t$,令$X_t=X_t^*-\mu_t$,其中$X_j^*= 0,\; j\le 0$,均值$\mu_t=E(X_t^*)$。时间序列向量$X_t$遵循{\bfseries{VARMA}}表示\parencite{escribano1994cointegration}
%	\begin{align}
%	\phi(B)X_t=\theta_B\varepsilon_t
%	\end{align}
%	其中
%	\begin{align}
%	\phi(B)&=I-\phi_1B-\cdots-\phi_pB_p\\
%	\theta(B)&=I-\theta_1B-\cdots-\theta_qB^q
%	\end{align}
%	$\phi_i, i=1,\ldots, p$和$\theta_i,i=1,\ldots, q$为$n\times n$的方阵,$B$是推移算子。
%
%\end{frame}
\begin{frame}{动态因子模型}
	令$x_{it}$为在时间$t=1,\ldots, T$第$i=1,\ldots,N$个观测值。因子模型由下式给出\parencite{breitung2011gls}
	$$
	x_{it}=\boldsymbol{\lambda}_i^{\prime}F_t+e_{it}
	$$
	其中$F_t=[f_{1t},\ldots,f_{rt}]'$是$r$维公共因子的向量,$\boldsymbol{\lambda}_i$是对应的$r$维因子载荷向量。
	用矩阵表示,模型可写成
	\begin{equation}
	\mathbf X = \mathbf F\bm \Lambda^{\prime}+\mathbf e
	\end{equation}

	
%	$y_t$为单时间序列变量,$X_t$是$N\times 1$维时间序列向量
%	\begin{align}
%	X_t& = Af_t + u_t\\
%	\phi(B)f_t& = \theta(B)a_t
%	\end{align}
%	其中 $A$是$n\times k$的因子载荷矩阵,$f_t$是公共因子向量,$a_t$和$u_t$是不相关的零均的白噪声过程,协方差矩阵分别为$\Sigma_u$和$\Sigma_a$。
\end{frame}


%
\begin{frame}{动态因子模型}
		其中
		\begin{itemize}
			\item $\mathbf X=[X_1,\ldots, X_T]'$是$T\times N$的观测矩阵,行向量为$X_t'=[x_{1t}, \ldots, x_{Nt}]$;
			\item $\mathbf e = [\mathbf e_1,\ldots,\mathbf e_T]'$是$T\times N$的特质(idiosyncratic)误差矩阵,行向量为$\mathbf e_t'=[e_{1t},\ldots, e_{Nt}]$;
			\item $\mathbf F=[ F_1, \ldots, F_T]'$
			\item $\bm \Lambda=[\boldsymbol{\lambda}_1,\ldots,\boldsymbol{\lambda}_N]'$
		\end{itemize}	
\end{frame}

\begin{frame}{预测问题}
	用时间序列向量$X_t$对单时间序列变量$y_t$进行预测。一般地,预测问题可以分两步求解\parencite{stock2002forecasting}
	\\~\\
	\begin{enumerate}
		\setlength\itemsep{1em}
		\item 根据$X_t$估计出因子$F_t$;
		\item 估计$y_t$和因子$F_t$之间的关系。
	\end{enumerate}
\end{frame}

%\begin{frame}{动态因子模型}
%	,$X_t$是$N\times 1$维时间序列向量。假设$(X_t, y_{t+h})$存在因子模型表示
%	\begin{equation}
%	X_t=\Lambda F_t+e_t
%	\end{equation}
%	且
%	\begin{equation}
%	y_{t+h}=\beta_F'F_t+\beta_w'w_t+\varepsilon_{t+h}
%	\end{equation}
%	其中$e_t$为$N\times 1$维误差向量,$w_t$为$m\times 1$维观测值,$\varepsilon_{t+h}$为预测误差。已有数据为$\{y_t, X_t, w_t\}_{t=1}^T$,目标是预测$y_{T+h}$
%\end{frame}


\section{方法综述}
%\begin{frame}{一般步骤}
%	一般地,预测问题可以分为两步\parencite{stock2002forecasting}
%	\begin{enumerate}
%		\item 根据$X_t$估计出因子;
%		\item 通过回归估计出响应变量和因子之间的关系。
%	\end{enumerate}
%\end{frame}
%
%\begin{frame}{N很小}
%	一般采取三步进行估计\parencite{bibid}
%	\begin{enumerate}
%		\item 假设$\{y_{t+h},X_t,w_t,e_t\}$联合随机过程的参数模型,用样本数据$\{y_{t+h}, X_t,w_t\}_{t=1}^{T-h}$估计模型中的参数;
%		\item 采用信号提取算法估计未知的因子
%		\item 用估计好的因子来预测$y_{T+h}$
%	\end{enumerate}
%\end{frame}
%\begin{frame}{N很大时的问题}
%	需要估计的参数太多了
%\end{frame}

%\begin{frame}{主成分估计}
%	\parencite{stock2002forecasting}提出主成分估计
%	考虑非线性最小二乘目标函数
%	$$
%	V(\tilde{F},\tilde{\lambda})=(NT)^{-1}\sum_{i}\sum_t(x_{it}-\tilde{\lambda_i}\tilde{F_t})^2
%	$$
%	写成矩阵形式是满足$T^{-1}\mathbf{FF'}=\mathbf I_r$约束下,最小化
%	$$
%	\mathrm{tr}[(\mathbf{X- \boldsymbol{\Lambda} F})'(\mathbf{X- \boldsymbol{\Lambda} F})]
%	$$
%	最后得到$F$的主成分估计为
%	$$
%	\tilde{F}=X'\hat{\Lambda}/N
%	$$
%	其中$\hat{\Lambda}$是$X'X$的前$r$大特征值对应的特征向量。
%\end{frame}
%
%
%\begin{frame}{GLS估计}
%	\begin{equation}
%	S(\mathbf F,\boldsymbol\Lambda,\boldsymbol\Omega)=\mathrm{tr}[\boldsymbol{\Omega}^{-1}(\mathbf{X- \boldsymbol{\Lambda} F})'(\mathbf{X- \boldsymbol{\Lambda} F})]
%	\end{equation}
%\end{frame}

\begin{frame}{主成分估计}
%	\citet{stock2002forecasting}提出主成分估计因子$F_t$,并进一步对$y_t$进行预测。具体地,
	在$T^{-1}\mathbf{FF'}=\mathbf I_r$约束下以及一定的假设条件下,最小化
	\begin{equation}
	S(\mathbf F,\boldsymbol{\Lambda})=\mathrm{tr}[(\mathbf{X- F\boldsymbol{\Lambda} }')'(\mathbf{X- F\boldsymbol{\Lambda} }')]
	\end{equation}
	得到估计
	$$
	\hat{\mathbf F}=\sqrt{T}\hat{\mathbf V}_r
	$$
	其中$\hat{\mathbf V}_r$是$\mathbf{XX'}$的前$r$个最大的特征值对应的特征向量组成的矩阵。\parencite{stock2002forecasting}
	
	
\end{frame}
\begin{frame}{GLS估计}
	考虑异方差,提出GLS估计,对加权残差平方和进行最小化
	\begin{equation}
	S(\mathbf F,\boldsymbol\Lambda,\boldsymbol\Omega)=\mathrm{tr}[(\mathbf{X- F\boldsymbol{\Lambda} }')'\boldsymbol{\Omega}^{-1}(\mathbf{X- F\boldsymbol{\Lambda} }')]
	\end{equation}
	根据$\boldsymbol{\Omega}$的不同,衍生了不同的方法
	\begin{enumerate}
		\setlength\itemsep{0.5em}
		\item $\boldsymbol{\Omega}=\mathrm{diag}[E(e_{1t}^2),\ldots,E(e_{Nt}^2)]$\parencite{choi2012efficient}
		\item 任意阵(\cite{forni2005generalized}; \cite{choi2012efficient})
		\item $\boldsymbol{\Omega}=\sigma^2(\mathbf I_N-\varrho\mathbf W_N)^{-1}(\mathbf I_N-\varrho\mathbf W_N')^{-1}$\parencite{ECTJ:ECTJ330}
		\item 同时考虑异方差和自相关性\parencite{breitung2011gls}
	\end{enumerate}
\end{frame}
\begin{frame}{小结}
	\begin{enumerate}
		\setlength\itemsep{1.5em}
		\item 很多论文研究对象是时间序列向量$X_t$(没有响应变量$y_t$),单纯考虑根据$X_t$来估计$F_t$,如\cite{breitung2011gls}
		\item
		即使论文研究对象是对$y_t$进行预测,但也是先根据$X_t$估计$F_t$,再由$F_t$预测$y_t$,如\cite{stock2002forecasting}
	\end{enumerate}
		\vspace{0.8em}
	\onslide<2->{\cjkhwxk 但如果我们提前用到$y_t$的信息呢?}
\end{frame}
\section{研究内容}
\begin{frame}{基本想法}
	处理高维数据时,首先根据$y_t$与$X_t$中每个时间序列的关系,提取其中与$y_t$相关性较大的时间序列构造新的时间序列向量$X_{t}^*$,对这个新的序列应用PC估计或者GLS估计。即
		\vspace{0.8em}
		\begin{enumerate}
			\item {\cjkhwxk 根据$y_t$和$X_t$得到初步降维后的$X_t^*$}
			\item 根据$X^*_t$估计出因子$F_t$;
			\item 估计$y_t$和因子$F_t$之间的关系。
		\end{enumerate}
	\vspace{0.8em}
	这个想法是受监督主成分方法(Supervised Principal Components)\parencite{bair2006prediction}的启发。
\end{frame}
\begin{frame}{监督主成分}
	
%	\parencite{bair2006prediction}提出监督主成分,其与主成分的区别时,首先对每个特征进行单变量回归,设定阈值$\theta$,选取那些系数大于阈值的特征构成降维后的矩阵,最后用降维后的矩阵进行普通的主成分。其相对主成分的优点,适用于高维数据。
%	
%	受监督主成分的启发,在建立动态因子模型来预测时,进行\parencite{stock2002forecasting}的主成分法前,首先预处理过滤掉多余的时间序列。
监督主成分的一般算法如下\parencite{bair2006prediction}
\vspace{0.5em}
\begin{enumerate}
	\setlength\itemsep{0.8em}
	\item 分别计算每个特征与输出变量之间的单变量回归系数;
	\item 从$0\le \theta_1\le\cdots\le \theta_K$中依次取阈值$\theta$
	\begin{enumerate}[a]
		\item 提取出原特征矩阵中单变量回归系数的绝对值大于$\theta$的特征,从新特征中取前$m$个主成分
		\item 采用这些主成分对输出变量进行预测
	\end{enumerate}
	\item 通过交叉验证选取$\theta$(和$m$)
\end{enumerate}
\vspace{1.2em}
\onslide<2->{\cjkhwxk 后续研究中,在第2(b)步中,尝试除PC之外的方法,如GLS。}
\end{frame}


\begin{frame}{研究计划}
	\begin{enumerate}
		\setlength\itemsep{1.2em}
		\item 推导有监督的方法(如有监督的主成分和GLS)应用于动态因子模型的公式,并给出具体的假设条件,同时尝试讨论其渐近性质;
		\item 通过模拟实验将之与PC估计、GLS估计进行比较;
		\item 将方法应用于实际数据,如股票数据。
	\end{enumerate}
\end{frame}


\begin{frame}[t,allowframebreaks]
	\frametitle{参考文献}
	\printbibliography[title=参考文献]
\end{frame}
\begin{frame}
	\chuhao\fontspec{LHANDW.TTF} {Thank you!}
\end{frame}
\end{document}


